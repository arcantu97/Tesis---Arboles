%Resumen

\chapter{Resumen}
\markboth{Resumen}{}

{\setlength{\leftskip}{10mm}
\setlength{\parindent}{-10mm}

\autor.

Candidato para obtener el grado de \grado\orientacion.

\uanl.

\fime.

Título del estudio: \textsc{\titulo}.

\noindent Número de páginas: \pageref*{lastpage}.}

%%% Comienza a llenar aquí
\paragraph{Objetivos y método de estudio:}
El objetivo de la investigación es realizar inventarios forestales por medio de las muestras recolectadas en el área del Cilantrillo y Trinidad. En la investigación se hace énfasis en utilizar la aplicación del procesamiento de imágenes para generar el inventario forestal.

\paragraph{Contribuciones y conclusiones:}
Durante la investigación se explica el funcionamiento del algoritmo que generará inventarios forestales usando el procesamiento de imágenes. Este algoritmo propone una solución eficiente y menos costosa en comparación a las técnicas manuales. 

\bigskip\noindent\begin{tabular}{lc}
\vspace*{-2mm}\hspace*{-2mm}Firma de la asesora: & \\
\cline{2-2} & \hspace*{1em}\asesor\hspace*{1em}
\end{tabular}


