%Resumen

\chapter{Resumen}
\markboth{Resumen}{}

{\setlength{\leftskip}{10mm}
\setlength{\parindent}{-10mm}

\autor.

Candidato para obtener el grado de \grado\orientacion.

\uanl.

\fime.

Título del estudio: \textsc{\titulo}.

\noindent Número de páginas: \pageref*{lastpage}.}

%%% Comienza a llenar aquí
\paragraph{Objetivos y método de estudio:}
El objetivo de la investigación es generar inventarios forestales por medio de muestras recolectadas en el área del Cilantrillo y Trinidad a través del procesamiento de imágenes, la visión computacional y el aprendizaje máquina, donde a partir de un modelo generado por el entrenamiento, se pueda detectar y marcar por color, cada especie de arból detectada en una muestra.

El método de estudio pretende entender la importancia de la visión computacional y el aprendizaje máquina en sectores que tienen otras finalidades, como lo son las ciencias forestales, que es donde se realizan inventarios forestales, cómo generar un inventario forestal y  qué clase de muestras son las que pueden ser utilizadas para generar un inventario forestal.
\paragraph{Contribuciones y conclusiones:}
Durante la investigación se explica el funcionamiento del algoritmo que generará inventarios forestales usando el procesamiento de imágenes. Este algoritmo propone una solución eficiente y menos costosa en comparación a las técnicas manuales. La solución propuesta está compuesta de un algoritmo de seis fases en las cuales se trata a las muestras para un propósito distinto en cada una de ellas. 

Además de proponer una solución eficiente, esta solución ahorra tiempos tanto de procesamiento, así como los tiempos utilizados para capturar muestras en las zonas que se pretende generar el inventario forestal. Esto último está solventado por el mejor umbral obtenido, que ayuda a poder determinar la mayor cantidad de especies en una región o área en la que se desea generar el inventario forestal.

La solución propuesta concluye en que el algoritmo obtiene un mejor rendimiento si se utiliza la cantidad total de muestras generadas por especie de árbol, además de esto, si se combina con el mejor porcentaje de umbralización  probado y un porcentaje adecuado de píxeles admitidos, permitirá detectar la mayor cantidad de especies posibles considerando las muestras recolectadas.

\bigskip\noindent\begin{tabular}{lc}
\vspace*{-2mm}\hspace*{-2mm}Firma de la asesora: & \\
\cline{2-2} & \hspace*{1em}\asesor\hspace*{1em}
\end{tabular}


